% Options for packages loaded elsewhere
\PassOptionsToPackage{unicode}{hyperref}
\PassOptionsToPackage{hyphens}{url}
%
\documentclass[
]{article}
\usepackage{amsmath,amssymb}
\usepackage{lmodern}
\usepackage{iftex}
\ifPDFTeX
  \usepackage[T1]{fontenc}
  \usepackage[utf8]{inputenc}
  \usepackage{textcomp} % provide euro and other symbols
\else % if luatex or xetex
  \usepackage{unicode-math}
  \defaultfontfeatures{Scale=MatchLowercase}
  \defaultfontfeatures[\rmfamily]{Ligatures=TeX,Scale=1}
\fi
% Use upquote if available, for straight quotes in verbatim environments
\IfFileExists{upquote.sty}{\usepackage{upquote}}{}
\IfFileExists{microtype.sty}{% use microtype if available
  \usepackage[]{microtype}
  \UseMicrotypeSet[protrusion]{basicmath} % disable protrusion for tt fonts
}{}
\makeatletter
\@ifundefined{KOMAClassName}{% if non-KOMA class
  \IfFileExists{parskip.sty}{%
    \usepackage{parskip}
  }{% else
    \setlength{\parindent}{0pt}
    \setlength{\parskip}{6pt plus 2pt minus 1pt}}
}{% if KOMA class
  \KOMAoptions{parskip=half}}
\makeatother
\usepackage{xcolor}
\IfFileExists{xurl.sty}{\usepackage{xurl}}{} % add URL line breaks if available
\IfFileExists{bookmark.sty}{\usepackage{bookmark}}{\usepackage{hyperref}}
\hypersetup{
  pdftitle={Bandersnatch VRF-AD Specification},
  pdfauthor={Davide Galassi; Seyed Hosseini},
  hidelinks,
  pdfcreator={LaTeX via pandoc}}
\urlstyle{same} % disable monospaced font for URLs
\setlength{\emergencystretch}{3em} % prevent overfull lines
\providecommand{\tightlist}{%
  \setlength{\itemsep}{0pt}\setlength{\parskip}{0pt}}
\setcounter{secnumdepth}{-\maxdimen} % remove section numbering
\newlength{\cslhangindent}
\setlength{\cslhangindent}{1.5em}
\newlength{\csllabelwidth}
\setlength{\csllabelwidth}{3em}
\newlength{\cslentryspacingunit} % times entry-spacing
\setlength{\cslentryspacingunit}{\parskip}
\newenvironment{CSLReferences}[2] % #1 hanging-ident, #2 entry spacing
 {% don't indent paragraphs
  \setlength{\parindent}{0pt}
  % turn on hanging indent if param 1 is 1
  \ifodd #1
  \let\oldpar\par
  \def\par{\hangindent=\cslhangindent\oldpar}
  \fi
  % set entry spacing
  \setlength{\parskip}{#2\cslentryspacingunit}
 }%
 {}
\usepackage{calc}
\newcommand{\CSLBlock}[1]{#1\hfill\break}
\newcommand{\CSLLeftMargin}[1]{\parbox[t]{\csllabelwidth}{#1}}
\newcommand{\CSLRightInline}[1]{\parbox[t]{\linewidth - \csllabelwidth}{#1}\break}
\newcommand{\CSLIndent}[1]{\hspace{\cslhangindent}#1}
\ifLuaTeX
  \usepackage{selnolig}  % disable illegal ligatures
\fi

\title{Bandersnatch VRF-AD Specification}
\author{Davide Galassi \and Seyed Hosseini}
\date{21 June 2024 - Draft 6}

\begin{document}
\maketitle

\newcommand{\G}{\langle G \rangle}
\newcommand{\F}{\mathbb{Z}^*_r}

\begin{center}\rule{0.5\linewidth}{0.5pt}\end{center}

\hypertarget{abstract}{%
\section{\texorpdfstring{\emph{Abstract}}{Abstract}}\label{abstract}}

This specification delineates the framework for a Verifiable Random
Function with Additional Data (VRF-AD), a cryptographic construct that
augments a standard VRF by incorporating auxiliary information into its
signature. We're going to first provide a specification to extend IETF's
ECVRF as outlined in
\href{https://datatracker.ietf.org/doc/rfc9381}{RFC-9381} {[}1{]}, then
we describe a variant of the Pedersen VRF originally introduced by
\href{https://eprint.iacr.org/2023/002}{BCHSV23} {[}2{]}, which serves
as a fundamental component for implementing anonymized ring signatures
as further elaborated by
\href{https://hackmd.io/ulW5nFFpTwClHsD0kusJAA}{Vasilyev} {[}3{]}. This
specification provides detailed insights into the usage of these
primitives with Bandersnatch, an elliptic curve constructed over the
BLS12-381 scalar field specified in
\href{https://eprint.iacr.org/2021/1152}{MSZ21} {[}4{]}.

\hypertarget{preliminaries}{%
\section{1. Preliminaries}\label{preliminaries}}

\textbf{Definition}: A \emph{verifiable random function with additional
data (VRF-AD)} can be described with two functions:

\begin{itemize}
\item
  \(Prove(sk,in,ad) \mapsto (out,\pi)\) : from secret key \(sk\), input
  \(in\), and additional data \(ad\) returns a verifiable output \(out\)
  and proof \(\pi\).
\item
  \(Verify(pk,in,ad,out,\pi) \mapsto (0|1)\) : for public key \(pk\),
  input \(in\), additional data \(ad\), output \(out\) and proof \(\pi\)
  returns either \(1\) on success or \(0\) on failure.
\end{itemize}

\hypertarget{vrf-input}{%
\subsection{1.1. VRF Input}\label{vrf-input}}

An arbitrary length octet-string provided by the user and used to
generate some unbiasable verifiable random output.

\hypertarget{vrf-input-point}{%
\subsection{1.2. VRF Input Point}\label{vrf-input-point}}

A point in \(\langle G \rangle\) generated from VRF input octet-string
using the \emph{Elligator 2} \emph{hash-to-curve} algorithm as described
by section 6.8.2 of
\href{https://datatracker.ietf.org/doc/rfc9380}{RFC-9380} {[}5{]}.

\hypertarget{vrf-output-point}{%
\subsection{1.3. VRF Output Point}\label{vrf-output-point}}

A point in \(\langle G \rangle\) generated from VRF input point as:
\(Output \leftarrow sk \cdot Input\).

\hypertarget{vrf-output}{%
\subsection{1.4. VRF Output}\label{vrf-output}}

A fixed length octet-string generated from VRF output point using the
proof-to-hash procedure defined in section 5.2 of
\href{https://datatracker.ietf.org/doc/rfc9381}{RFC-9381}.

\hypertarget{additional-data}{%
\subsection{1.5 Additional Data}\label{additional-data}}

An arbitrary length octet-string provided by the user to be signed
together with the generated VRF output. This data doesn't influence the
produced VRF output.

\hypertarget{ietf-vrf}{%
\section{2. IETF VRF}\label{ietf-vrf}}

Based on IETF \href{https://datatracker.ietf.org/doc/rfc9381}{RFC-9381}
which is extended with the capability to sign additional user data
(\texttt{ad}).

\hypertarget{configuration}{%
\subsection{2.1. Configuration}\label{configuration}}

Configuration is given by following the \emph{``cipher suite''}
guidelines defined in section 5.5 of
\href{https://datatracker.ietf.org/doc/rfc9381}{RFC-9381}.

\begin{itemize}
\item
  \texttt{suite\_string} = \texttt{"Bandersnatch\_SHA-512\_ELL2"}.
\item
  The EC group \(\langle G \rangle\) is the prime subgroup of the
  Bandersnatch elliptic curve, in Twisted Edwards form, with finite
  field and curve parameters as specified in
  \href{https://eprint.iacr.org/2021/1152}{MSZ21}. For this group,
  \texttt{fLen} = \texttt{qLen} = \(32\) and \texttt{cofactor} = \(4\).
\item
  The prime subgroup generator \(G \in \langle G \rangle\) is defined as
  follows:
  \[_{G.x = \texttt{0x29c132cc2c0b34c5743711777bbe42f32b79c022ad998465e1e71866a252ae18}}\]
  \[_{G.y = \texttt{0x2a6c669eda123e0f157d8b50badcd586358cad81eee464605e3167b6cc974166}}\]
\item
  \texttt{cLen} = 32.
\item
  The public key generation primitive is \(pk = sk \cdot G\), with
  \(sk\) the secret key scalar and \(G\) the group generator. In this
  cipher suite, the secret scalar \texttt{x} is equal to the secret key
  \texttt{sk}.
\item
  \texttt{encode\_to\_curve\_salt} = \texttt{pk\_string}
  (i.e.~\texttt{point\_to\_string(pk)}).
\item
  The \texttt{ECVRF\_nonce\_generation} function is specified in section
  5.4.2.1 of \href{https://datatracker.ietf.org/doc/rfc9381}{RFC-9381}.
\item
  The \texttt{int\_to\_string} function encodes into the 32 bytes little
  endian representation.
\item
  The \texttt{string\_to\_int} function decodes from the 32 bytes little
  endian representation eventually reducing modulo the prime field
  order.
\item
  The \texttt{point\_to\_string} function converts a point in
  \(\langle G \rangle\) to an octet-string using compressed form. The
  \(y\) coordinate is encoded using \texttt{int\_to\_string} function
  and the most significant bit of the last octet is used to keep track
  of \(x\) sign. This implies that \texttt{ptLen\ =\ flen\ =\ 32}.
\item
  The \texttt{string\_to\_point} function converts an octet-string to a
  point on \(E\). The string most significant bit is removed to recover
  the \(x\) coordinate as function of \(y\), which is first decoded from
  the rest of the string using \texttt{int\_to\_string} procedure. This
  function MUST outputs ``INVALID'' if the octet-string does not decode
  to a point on the prime subgroup \(\langle G \rangle\).
\item
  The hash function \texttt{hash} is SHA-512 as specified in
  \href{https://datatracker.ietf.org/doc/rfc6234}{RFC-6234} {[}6{]},
  with \texttt{hLen} = 64.
\item
  The \texttt{ECVRF\_encode\_to\_curve} function uses \emph{Elligator2}
  method described in section 6.8.2 of
  \href{https://datatracker.ietf.org/doc/rfc9380}{RFC-9380} and is
  described in section 5.4.1.2 of
  \href{https://datatracker.ietf.org/doc/rfc9381}{RFC-9381}, with
  \texttt{h2c\_suite\_ID\_string} =
  \texttt{"Bandersnatch\_XMD:SHA-512\_ELL2\_RO\_"} and domain separation
  tag \texttt{DST\ =\ "ECVRF\_"} \(\Vert\)
  \texttt{h2c\_suite\_ID\_string} \(\Vert\) \texttt{suite\_string}.
\end{itemize}

\hypertarget{prove}{%
\subsection{2.2. Prove}\label{prove}}

\textbf{Input}:

\begin{itemize}
\tightlist
\item
  \(x \in \mathbb{Z}^*_r\): Secret key
\item
  \(I \in \langle G \rangle\): VRF input point
\item
  \(ad\): Additional data octet-string
\end{itemize}

\textbf{Output}:

\begin{itemize}
\tightlist
\item
  \(O \in \langle G \rangle\): VRF output point
\item
  \(\pi \in (\mathbb{Z}^*_r, \mathbb{Z}^*_r)\): Schnorr-like proof
\end{itemize}

\textbf{Steps}:

\begin{enumerate}
\def\labelenumi{\arabic{enumi}.}
\tightlist
\item
  \(O \leftarrow x \cdot I\)
\item
  \(Y \leftarrow x \cdot G\)
\item
  \(k \leftarrow nonce(x, I)\)
\item
  \(c \leftarrow challenge(Y, I, O, k \cdot G, k \cdot I, ad)\)
\item
  \(s \leftarrow k + c \cdot x\)
\item
  \(\pi \leftarrow (c, s)\)
\item
  \textbf{return} \((O, \pi)\)
\end{enumerate}

\textbf{Externals}:

\begin{itemize}
\tightlist
\item
  \(nonce\): refer to section 5.4.2.1 of
  \href{https://datatracker.ietf.org/doc/rfc9381}{RFC-9381}.
\item
  \(challenge\): refer to section 5.4.3 of
  \href{https://datatracker.ietf.org/doc/rfc9381}{RFC-9381} and section
  2.4 of this specification.
\end{itemize}

\hypertarget{verify}{%
\subsection{2.3. Verify}\label{verify}}

\textbf{Input}:

\begin{itemize}
\tightlist
\item
  \(Y \in \langle G \rangle\): Public key
\item
  \(I \in \langle G \rangle\): VRF input point
\item
  \(ad\): Additional data octet-string
\item
  \(O \in \langle G \rangle\): VRF output point
\item
  \(\pi \in (\mathbb{Z}^*_r, \mathbb{Z}^*_r)\): Schnorr-like proof
\end{itemize}

\textbf{Output}:

\begin{itemize}
\tightlist
\item
  True if proof is valid, False otherwise
\end{itemize}

\textbf{Steps}:

\begin{enumerate}
\def\labelenumi{\arabic{enumi}.}
\tightlist
\item
  \((c, s) \leftarrow \pi\)
\item
  \(U \leftarrow s \cdot G - c \cdot Y\)
\item
  \(V \leftarrow s \cdot I - c \cdot O\)
\item
  \(c' \leftarrow challenge(Y, I, O, U, V, ad)\)
\item
  \textbf{if} \(c \neq c'\) \textbf{then} \textbf{return} False
\item
  \textbf{return} True
\end{enumerate}

\textbf{Externals}:

\begin{itemize}
\tightlist
\item
  \(challenge\): as defined for \(Sign\)
\end{itemize}

\hypertarget{challenge}{%
\subsection{2.4. Challenge}\label{challenge}}

Challenge construction mostly follows the procedure given in section
5.4.3 of \href{https://datatracker.ietf.org/doc/rfc9381}{RFC-9381}
{[}1{]} with some tweaks to add additional data.

\textbf{Input}:

\begin{itemize}
\tightlist
\item
  \(Points \in \langle G \rangle^n\): Sequence of \(n\) points.
\item
  \(ad\): Additional data octet-string
\end{itemize}

\textbf{Output}:

\begin{itemize}
\tightlist
\item
  \(c \in \mathbb{Z}^*_r\): Challenge scalar.
\end{itemize}

\textbf{Steps}:

\begin{enumerate}
\def\labelenumi{\arabic{enumi}.}
\tightlist
\item
  \(str\) = \texttt{suite\_string} \(\Vert\) \texttt{0x02}
\item
  \textbf{for each} \(P\) \textbf{in} \(Points\): \(str = str \Vert\)
  \texttt{point\_to\_string(}\(P\)\texttt{)}\$
\item
  \(str = str \Vert ad \Vert 0x00\)
\item
  \(h =\) \texttt{hash(}\(str\)\texttt{)}
\item
  \(h_t = h[0] \Vert .. \Vert h[cLen - 1]\)
\item
  \(c =\) \texttt{string\_to\_int(}\(h_t\)\texttt{)}
\item
  \textbf{return} \(c\)
\end{enumerate}

With \texttt{point\_to\_string}, \texttt{string\_to\_int} and
\texttt{hash} as defined in section 2.1.

\hypertarget{pedersen-vrf}{%
\section{3. Pedersen VRF}\label{pedersen-vrf}}

Pedersen VRF resembles IETF EC-VRF but replaces the public key with a
Pedersen commitment to the secret key, which makes this VRF useful in
anonymized ring proofs.

The scheme proves that the output has been generated with a secret key
associated with a blinded public key (instead of the public key). The
blinded public key is a cryptographic commitment to the public key, and
it can be unblinded to prove that the output of the VRF corresponds to
the public key of the signer.

This specification mostly follows the design proposed by
\href{https://eprint.iacr.org/2023/002}{BCHSV23} {[}2{]} in section 4
with some details about blinding base point value and challenge
generation procedure.

\hypertarget{configuration-1}{%
\subsection{3.1. Configuration}\label{configuration-1}}

Pedersen VRF is configured for prime subgroup \(\langle G \rangle\) of
Bandersnatch elliptic curve \(E\) defined in
\href{https://eprint.iacr.org/2021/1152}{MSZ21} {[}4{]} with
\emph{blinding base} \(B \in \langle G \rangle\) defined as follows:

\[_{B.x = \texttt{0x2039d9bf2ecb2d4433182d4a940ec78d34f9d19ec0d875703d4d04a168ec241e}}\]
\[_{B.y = \texttt{0x54fa7fd5193611992188139d20221028bf03ee23202d9706a46f12b3f3605faa}}\]

For all the other configurable parameters and external functions we
adhere as much as possible to the Bandersnatch cipher suite for IETF VRF
described in section 2.1 of this specification.

\hypertarget{prove-1}{%
\subsubsection{3.2. Prove}\label{prove-1}}

\textbf{Input}:

\begin{itemize}
\tightlist
\item
  \(x \in \mathbb{Z}^*_r\): Secret key
\item
  \(b \in \mathbb{Z}^*_r\): Secret blinding factor
\item
  \(I \in \langle G \rangle\): VRF input point
\item
  \(ad\): Additional data octet-string
\end{itemize}

\textbf{Output}:

\begin{itemize}
\tightlist
\item
  \(O \in \langle G \rangle\): VRF output point
\item
  \(\pi \in (\langle G \rangle, \langle G \rangle, \langle G \rangle, \mathbb{Z}^*_r, \mathbb{Z}^*_r)\):
  Pedersen proof
\end{itemize}

\textbf{Steps}:

\begin{enumerate}
\def\labelenumi{\arabic{enumi}.}
\tightlist
\item
  \(O \leftarrow x \cdot I\)
\item
  \(k \leftarrow nonce(x, I)\)
\item
  \(k_b \leftarrow nonce(k, I)\)
\item
  \(\bar{Y} \leftarrow x \cdot G + b \cdot B\)
\item
  \(R \leftarrow k \cdot G + k_b \cdot B\)
\item
  \(O_k \leftarrow k \cdot I\)
\item
  \(c \leftarrow challenge(\bar{Y}, I, O, R, O_k, ad)\)
\item
  \(s \leftarrow k + c \cdot x\)
\item
  \(s_b \leftarrow k_b + c \cdot b\)
\item
  \(\pi \leftarrow (\bar{Y}, R, O_k, s, s_b)\)
\item
  \textbf{return} \((O, \pi)\)
\end{enumerate}

\hypertarget{verify-1}{%
\subsection{3.3. Verify}\label{verify-1}}

\textbf{Input}:

\begin{itemize}
\tightlist
\item
  \(I \in \langle G \rangle\): VRF input point
\item
  \(ad\): Additional data octet-string
\item
  \(O \in \langle G \rangle\): VRF output point
\item
  \(\pi \in (\langle G \rangle, \langle G \rangle, \langle G \rangle, \mathbb{Z}^*_r, \mathbb{Z}^*_r)\):
  Pedersen proof
\end{itemize}

\textbf{Output}:

\begin{itemize}
\tightlist
\item
  True if proof is valid, False otherwise
\end{itemize}

\textbf{Steps}:

\begin{enumerate}
\def\labelenumi{\arabic{enumi}.}
\tightlist
\item
  \((\bar{Y}, R, O_k, s, s_b) \leftarrow \pi\)
\item
  \(c \leftarrow challenge(\bar{Y}, I, O, R, O_k, ad)\)
\item
  \textbf{if} \(O_k + c \cdot O \neq I \cdot s\) \textbf{then}
  \textbf{return} False
\item
  \textbf{if} \(R + c \cdot \bar{Y} \neq s \cdot G - s_b \cdot B\)
  \textbf{then} \textbf{return} False
\item
  \textbf{return} True
\end{enumerate}

\hypertarget{ring-vrf}{%
\section{4. Ring VRF}\label{ring-vrf}}

Anonymized ring VRF based of {[}Pedersen VRF{]} and Ring Proof as
proposed by \href{https://hackmd.io/ulW5nFFpTwClHsD0kusJAA}{Vasilyev}.

\hypertarget{configuration-2}{%
\subsection{4.1. Configuration}\label{configuration-2}}

Setup for plain {[}Pedersen VRF{]} applies.

Ring proof configuration:

\begin{itemize}
\tightlist
\item
  KZG PCS uses
  \href{https://zfnd.org/conclusion-of-the-powers-of-tau-ceremony}{Zcash}
  SRS and a domain of 2048 entries.
\item
  \(G_1\): BLS12-381 \(G_1\)
\item
  \(G_2\): BLS12-381 \(G_2\)
\item
  TODO: \ldots{}
\end{itemize}

\hypertarget{prove-2}{%
\subsection{4.2. Prove}\label{prove-2}}

\textbf{Input}:

\begin{itemize}
\tightlist
\item
  \(x \in \mathbb{Z}^*_r\): Secret key
\item
  \(P \in TODO\): Ring prover
\item
  \(b \in \mathbb{Z}^*_r\): Secret blinding factor
\item
  \(I \in \langle G \rangle\): VRF input point
\item
  \(ad\): Additional data octet-string
\end{itemize}

\textbf{Output}:

\begin{itemize}
\tightlist
\item
  \(O \in \langle G \rangle\): VRF output point
\item
  \(\pi_p \in (\langle G \rangle, \langle G \rangle, \langle G \rangle, \mathbb{Z}^*_r, \mathbb{Z}^*_r)\):
  Pedersen proof
\item
  \(\pi_r \in ((G_1)^4, (\mathbb{Z}^*_r)^7, G_1, \mathbb{Z}^*_r, G_1, G_1)\):
  Ring proof
\end{itemize}

\textbf{Steps}:

\begin{enumerate}
\def\labelenumi{\arabic{enumi}.}
\tightlist
\item
  \((O, \pi_p) \leftarrow Pedersen.prove(x, b, I, ad)\)
\item
  \(\pi_r \leftarrow Ring.prove(P, b)\) (TODO)
\item
  \textbf{return} \((O, \pi_p, \pi_r)\)
\end{enumerate}

\hypertarget{verify-2}{%
\subsection{4.3. Verify}\label{verify-2}}

\textbf{Input}:

\begin{itemize}
\tightlist
\item
  \(V \in (G_1)^3\): Ring verifier
\item
  \(I \in \langle G \rangle\): VRF input point
\item
  \(O\): VRF Output \(\in \langle G \rangle\).
\item
  \(ad\): Additional data octet-string
\item
  \(\pi_p \in (\langle G \rangle, \langle G \rangle, \langle G \rangle, \mathbb{Z}^*_r, \mathbb{Z}^*_r)\):
  Pedersen proof
\item
  \(\pi_r \in ((G_1)^4, (\mathbb{Z}^*_r)^7, G_1, \mathbb{Z}^*_r, G_1, G_1)\):
  Ring proof
\end{itemize}

\textbf{Output}:

\begin{itemize}
\tightlist
\item
  True if proof is valid, False otherwise
\end{itemize}

\textbf{Steps}:

\begin{enumerate}
\def\labelenumi{\arabic{enumi}.}
\tightlist
\item
  \(rp = Pedersen.verify(I, ad, O, \pi_p)\)
\item
  \textbf{if} \(rp \neq True\) \textbf{return} False
\item
  \((\bar{Y}, R, O_k, s, s_b) \leftarrow \pi_p\)
\item
  \(rr = Ring.verify(V, \pi_r, \bar{Y})\)
\item
  \textbf{if} \(rr \neq True\) \textbf{return} False
\item
  \textbf{return} True
\end{enumerate}

\hypertarget{references}{%
\section*{5. References}\label{references}}
\addcontentsline{toc}{section}{5. References}

\hypertarget{refs}{}
\begin{CSLReferences}{0}{0}
\leavevmode\vadjust pre{\hypertarget{ref-RFC9381}{}}%
\CSLLeftMargin{1. }
\CSLRightInline{Internet Engineering Task Force
\emph{\href{https://datatracker.ietf.org/doc/rfc9381}{{Verifiable Random
Functions}}}; {RFC Editor}, 2023;}

\leavevmode\vadjust pre{\hypertarget{ref-BCHSV23}{}}%
\CSLLeftMargin{2. }
\CSLRightInline{Burdges, J.; Ciobotaru, O.; Alper, H.K.; Stewart, A.;
Vasilyev, S. \href{https://eprint.iacr.org/2023/002}{Ring Verifiable
Random Functions and Zero-Knowledge Continuations} 2023.}

\leavevmode\vadjust pre{\hypertarget{ref-Vasilyev}{}}%
\CSLLeftMargin{3. }
\CSLRightInline{Vasilyev, S.
\href{https://hackmd.io/ulW5nFFpTwClHsD0kusJAA}{Ring Proof Technical
Specification} 2024.}

\leavevmode\vadjust pre{\hypertarget{ref-MSZ21}{}}%
\CSLLeftMargin{4. }
\CSLRightInline{Masson, S.; Sanso, A.; Zhang, Z.
\href{https://eprint.iacr.org/2021/1152}{Bandersnatch: A Fast Elliptic
Curve Built over the Bls12-381 Scalar Field} 2021.}

\leavevmode\vadjust pre{\hypertarget{ref-RFC9380}{}}%
\CSLLeftMargin{5. }
\CSLRightInline{Internet Engineering Task Force
\emph{\href{https://datatracker.ietf.org/doc/rfc9380}{{Hashing to
Elliptic Curves}}}; {RFC Editor}, 2023;}

\leavevmode\vadjust pre{\hypertarget{ref-RFC6234}{}}%
\CSLLeftMargin{6. }
\CSLRightInline{Internet Engineering Task Force
\emph{\href{https://datatracker.ietf.org/doc/rfc6234}{{US Secure Hash
Algorithms}}}; {RFC Editor}, 2011;}

\end{CSLReferences}

\end{document}
